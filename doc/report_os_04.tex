\documentclass[12pt]{article}

\usepackage{fullpage}
\usepackage{multicol,multirow}
\usepackage{tabularx}
\usepackage{ulem}
\usepackage[utf8]{inputenc}
\usepackage[russian]{babel}
\usepackage{minted}

\usepackage{color} %% это для отображения цвета в коде
\usepackage{listings} %% собственно, это и есть пакет listings

\lstset{ %
  language=C,                 % выбор языка для подсветки (здесь это С)
  basicstyle=\small\sffamily, % размер и начертание шрифта для подсветки кода
  numbers=left,               % где поставить нумерацию строк (слева\справа)
  %numberstyle=\tiny,           % размер шрифта для номеров строк
  stepnumber=1,                   % размер шага между двумя номерами строк
  numbersep=5pt,                % как далеко отстоят номера строк от подсвечиваемого кода
  backgroundcolor=\color{white}, % цвет фона подсветки - используем \usepackage{color}
  showspaces=false,            % показывать или нет пробелы специальными отступами
  showstringspaces=false,      % показывать или нет пробелы в строках
  showtabs=false,             % показывать или нет табуляцию в строках
  frame=single,              % рисовать рамку вокруг кода
  tabsize=2,                 % размер табуляции по умолчанию равен 2 пробелам
  captionpos=t,              % позиция заголовка вверху [t] или внизу [b] 
  breaklines=true,           % автоматически переносить строки (да\нет)
  breakatwhitespace=false, % переносить строки только если есть пробел
  escapeinside={\%*}{*)}   % если нужно добавить комментарии в коде
}

\begin{document}
\begin{titlepage}
  \large
  \begin{center} 
    
      Московский Авиационный Интститут \\
      (Национальный Исследовательский Университет) \\
      Факультет информационных технологий и прикладной математики \\
      Кафедра вычислительной математики и программирования \\
      \vfill\vfill
      \textbf{
        { Лабораторная работа №4 по курсу} \\ 
        <<Операционные системы>> \\
        \bigskip
            {Файловые системы и ``File mapping'' } \\
    } 
  \end{center}
  \vfill

  \begin{flushright}

    Студент:  {Артемьев Дмитрий Иванович}

    Группа: {М8О-206Б-18}

    Вариант: {23}
    
    Преподаватель: {Соколов Андрей Алексеевич}

    Оценка: $\rule{3cm}{0.15mm}$

    Дата: $\rule{3cm}{0.15mm}$
    
    Подпись: $\rule{3cm}{0.15mm}$

  \end{flushright}
  \vfill
  \begin{center}
    Москва, 2019
  \end{center}
  
\end{titlepage}

\subsection*{Условие}

Составить и отладить программу на языке Си, осуществляющую работу с процессами и
взаимодействие между ними в одной из двух операционных систем. В результате работы
программа (основной процесс) должен создать для решение задачи один или несколько
дочерних процессов. Взаимодействие между процессами осуществляется через системные
сигналы/события и/или через отображаемые файлы (memory-mapped files).
Необходимо обрабатывать системные ошибки, которые могут возникнуть в результате работы.

23. Родительский процесс считывает две координаты передает их через канал дочернему
процессу. Дочерний процесс определяет к какой четверти относится точка, а далее передает
результат родительскому процессу.

\subsection*{Описание программы}

Код программы находится в файле main.c.

\subsection*{Ход выполнения программы}
\begin{enumerate}
\item Создание временного файла
\item Создание общей области с помощью mmap 
\item Создание двух семафоров
\item Создание дочернего процесса
\item Ожидание дочерним процессом первого семафора
\item Считывание из стандартного потока входных данных родительским процессом, запись в общую область памяти
\item Вызов первого семафора (дочернего процесса), ожидание второго семафора родительским процессом
\item Обработка данных из общей области памяти дочерним процессом, запись в неё же результата
\item Вызов второго семафора, завершение работы дочернего процесса
\item Считывание результата из общей области памяти родительским процессом, вывод результата на экран
\item Завершение работы родительского процесса
\end{enumerate}

\subsection*{Недочёты}



\subsection*{Выводы}

Я научился взаимодействию процессов посредством маппинга и семафоров. 
\pagebreak

\vfill

\subsection*{Исходный код}

{\Huge main.c}
\inputminted
    {C}{../src/main.c}
    \pagebreak    
\end{document}
